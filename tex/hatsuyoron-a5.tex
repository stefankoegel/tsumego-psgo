\documentclass[11pt]{book}

\usepackage[a5paper,margin=1.5cm]{geometry}
\usepackage{float}
\usepackage{caption}

\usepackage{multicol}
\setlength{\columnsep}{0.5cm}

\usepackage{psgo}
\setlength{\goxunit}{0.29cm}
\setlength{\goyunit}{0.29cm}

\pagestyle{plain}

% Format go problems
\newcommand{\goproblem}[2]{%
    \begin{figure}[H]
        \caption*{\it #1}
        \vspace{-1em}
        \centering
        #2
        \vspace{-1em}
    \end{figure}
}

\begin{document}

\begin{titlepage}

    \vspace*{0.2\textheight}

    \begin{center}
        \textit{\large [Missing kanji.pdf]}
    \end{center}
    \begin{center}
        \textbf{\huge igo hatsuyo-ron}
    \end{center}

\end{titlepage}

\newpage % ---------------------------------------------------------------------

\pagenumbering{gobble}

\noindent\textbf{\Large motto:}

\medskip
\textit{\normalsize ``It is a matter of life and death, a road either to safety or to ruin. Hence it is a subject of inquiry which can on no account be neglected.''}

\medskip
\hfill {\it Sun Tzu: The Art of War}
\bigskip

\noindent\textbf{\Large preface:}

\medskip
{
\normalsize
This problem collection was compiled around 1710 by Inoue Dosetsu Inseki, fourth head of the Inoue house and fifth Meijin Godokoro. It was designed for training of the best students of the Inoue school and was kept secret for a long time.

Igo Hatsuyo-ron consists of 183 mostly insanely difficult problems and is aimed at serious players with deep interest in the game. While solving the problems takes many months, possibly even years, finding the solution is always particularly rewarding.

I wish you enjoyment and improving in the wonderful game of go, weiqi, baduk, or whatever you like to call it.
}

\medskip
\hfill \textit{tasuki}

\hfill \textit{27.11.2006}

\newpage % ---------------------------------------------------------------------

\pagenumbering{arabic}

\begin{multicols}{2}
    \input{tmp/hatsuyoron-problems.tex}
\end{multicols}

\end{document}