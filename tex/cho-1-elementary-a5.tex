\documentclass[11pt]{book}

\usepackage[a5paper,margin=1.5cm]{geometry}
\usepackage{float}
\usepackage{caption}

\usepackage{multicol}
\setlength{\columnsep}{0.5cm}

\usepackage{psgo}
\setlength{\goxunit}{0.29cm}
\setlength{\goyunit}{0.29cm}

\pagestyle{plain}

% Format go problems
\newcommand{\goproblem}[2]{%
    \begin{figure}[H]
        \caption*{\it #1}
        \vspace{-1em}
        \centering
        #2
        \vspace{-1em}
    \end{figure}
}

\begin{document}

\begin{titlepage}

    \vspace*{0.2\textheight}

    \begin{center}
        \textit{\large Cho Chikun's}
    \end{center}
    \begin{center}
        \textbf{\huge encyclopedia of life and death}
    \end{center}
    \begin{center}
        {\large part first -- elementary problems}
    \end{center}

\end{titlepage}

\newpage % ---------------------------------------------------------------------

\pagenumbering{gobble}

\noindent\textbf{\Large motto:}

\medskip
\textit{\normalsize ``It is a matter of life and death, a road either to safety or to ruin. Hence it is a subject of inquiry which can on no account be neglected.''}

\medskip
\hfill {\it Sun Tzu: The Art of War}
\bigskip

\noindent\textbf{\Large preface:}

\medskip
{
\normalsize
This is a collection of almost three thousand problems from Encyclopedia of life and death by Cho Chikun. The problems come without solution because of two particular reasons: first, I think we can learn more by actually solving the problem, trying all the possible variations; and second, the solutions are copyrighted. It is always {\bf black to move}, so I only show pictures without any distracting text around.

In this first part, you can find about nine hundred problems mostly for low kyu players, but also dan players can benefit from going through lots of such problems very quickly -- dan player should get the solution in a few seconds, so he is likely to spend about an hour reading it through.

I wish you enjoyment and improving in the wonderful game of go, weiqi, baduk, or whatever you like to call it.
}

\medskip
\hfill \textit{tasuki}

\hfill \textit{27.11.2004}

\newpage % ---------------------------------------------------------------------

\pagenumbering{arabic}

\begin{multicols}{2}
    \input{tmp/cho-1-elementary-problems.tex}
\end{multicols}

\end{document}