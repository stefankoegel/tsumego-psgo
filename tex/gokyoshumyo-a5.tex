\documentclass[11pt]{book}

\usepackage[a5paper,margin=1.5cm]{geometry}
\usepackage{float}
\usepackage{caption}

\usepackage{multicol}
\setlength{\columnsep}{0.5cm}

\usepackage{psgo}
\setlength{\goxunit}{0.29cm}
\setlength{\goyunit}{0.29cm}

\pagestyle{plain}

% Format go problems
\newcommand{\goproblem}[2]{%
    \begin{figure}[H]
        \caption*{\it #1}
        \vspace{-1em}
        \centering
        #2
        \vspace{-1em}
    \end{figure}
}

\begin{document}

\begin{titlepage}

    \vspace*{0.2\textheight}

    \begin{center}
        \textit{\large [Missing gshumyo.pdf]}
    \end{center}
    \begin{center}
        \textbf{\huge gokyo shumyo}
    \end{center}

\end{titlepage}

\newpage % ---------------------------------------------------------------------

\pagenumbering{gobble}

\noindent\textbf{\Large motto:}

\medskip
\textit{\normalsize ``It is a matter of life and death, a road either to safety or to ruin. Hence it is a subject of inquiry which can on no account be neglected.''}

\medskip
\hfill {\it Sun Tzu: The Art of War}
\bigskip

\noindent\textbf{\Large preface:}

\medskip
{
\normalsize
Gokyo Shumyo is a classical problem collection which was published by Hayashi Genbi in 1822. Its name could be translated to English as ``Brilliances from go classics''. The collection consists of 520 problems, which are divided into following seven sections:

\begin{enumerate}
    \setlength{\itemsep}{0pt}
    \setlength{\parskip}{0pt}

    \item living (103 problems)\dotfill 1
    \item killing (71 problems)\dotfill 10
    \item creating a ko (90 problems)\dotfill 16 % I adjusted the page numbers by hand
    \item capturing races (96 problems)\dotfill 25
    \item oiotoshi (40 problems)\dotfill 34
    \item connecting (74 problems)\dotfill 37
    \item various techniques (46 problems)\dotfill 45
\end{enumerate}

Most of these problems should be solvable for almost everyone, but some are really difficult, so feel free to skip them if you can't find the solution.

I wish you to enjoy and improve while studying these problems.
}

\medskip
\hfill \textit{tasuki}

\hfill \textit{1.12.2006}

\newpage % ---------------------------------------------------------------------

\pagenumbering{arabic}

\begin{multicols}{2}
    \input{tmp/gokyoshumyo-problems.tex}
\end{multicols}

\end{document}